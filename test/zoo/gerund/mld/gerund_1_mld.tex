\documentclass[a4paper]{article}
\usepackage[normalem]{ulem}
\usepackage[T1]{fontenc}
\usepackage[french]{babel}
\frenchsetup{StandardLayout=true}

\newcommand{\relat}[1]{\textsc{#1}}
\newcommand{\attr}[1]{#1}
\newcommand{\prim}[1]{\uline{#1}}
\newcommand{\foreign}[1]{\#\textsl{#1}}

\title{Conversion en relationnel\\du MCD \emph{gerund}}
\author{\emph{Généré par Mocodo}}

\begin{document}
\maketitle

\begin{itemize}
  \item \relat{Commande} (\prim{commande}, \attr{date})
  \begin{itemize}
    \item Le champ \emph{commande} constitue la clé primaire de la table. C'était déjà un identifiant de l'entité \emph{Commande}.
    \item Le champ \emph{date} était déjà un simple attribut de l'entité \emph{Commande}.
  \end{itemize}

  \item \relat{Ligne de commande} (\foreign{\prim{commande}}, \foreign{\prim{produit}}, \attr{quantité})
  \begin{itemize}
    \item Le champ \emph{commande} fait partie de la clé primaire de la table. C'est une clé étrangère qui a migré directement à partir de l'entité \emph{Commande}.
    \item Le champ \emph{produit} fait partie de la clé primaire de la table. C'est une clé étrangère qui a migré directement à partir de l'entité \emph{Produit}.
    \item Le champ \emph{quantité} était déjà un simple attribut de l'entité \emph{Ligne de commande}.
  \end{itemize}

  \item \relat{Produit} (\prim{produit}, \attr{libellé})
  \begin{itemize}
    \item Le champ \emph{produit} constitue la clé primaire de la table. C'était déjà un identifiant de l'entité \emph{Produit}.
    \item Le champ \emph{libellé} était déjà un simple attribut de l'entité \emph{Produit}.
  \end{itemize}

\end{itemize}

\end{document}
