% Generated by Mocodo 4.0.0

\documentclass[a4paper]{article}
\usepackage[normalem]{ulem}
\usepackage[T1]{fontenc}
\usepackage[french]{babel}
\frenchsetup{StandardLayout=true}

\newcommand{\relat}[1]{\textsc{#1}}
\newcommand{\attr}[1]{#1}
\newcommand{\prim}[1]{\uline{#1}}
\newcommand{\foreign}[1]{\#\textsl{#1}}

\title{Conversion en relationnel\\du MCD \emph{reflexive}}
\author{\emph{Généré par Mocodo}}

\begin{document}
\maketitle

\begin{itemize}
  \item \relat{COMPOSER} (\prim{pièce}, \prim{pièce 1})
  \begin{itemize}
    \item Les champs \emph{pièce} et \emph{pièce 1} constituent la clé primaire de la table. Leur table d'origine (\emph{PIÈCE}) ayant été supprimée, ils ne sont pas considérés comme clés étrangères.
  \end{itemize}

  \item \relat{HOMME} (\prim{Num. SS}, \attr{Nom}, \attr{Prénom}, \foreign{Num. SS 1!})
  \begin{itemize}
    \item Le champ \emph{Num. SS} constitue la clé primaire de la table. C'était déjà un identifiant de l'entité \emph{HOMME}.
    \item Les champs \emph{Nom} et \emph{Prénom} étaient déjà de simples attributs de l'entité \emph{HOMME}.
    \item Le champ à saisie obligatoire \emph{Num. SS 1} est une clé étrangère. Il a migré par l'association de dépendance fonctionnelle \emph{ENGENDRER} à partir de l'entité \emph{HOMME} en perdant son caractère identifiant.
  \end{itemize}

\end{itemize}

\end{document}
