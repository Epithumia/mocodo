% Generated by Mocodo 4.0.0

\documentclass[a4paper]{article}
\usepackage[normalem]{ulem}
\usepackage[T1]{fontenc}
\usepackage[french]{babel}
\frenchsetup{StandardLayout=true}

\newcommand{\relat}[1]{\textsc{#1}}
\newcommand{\attr}[1]{#1}
\newcommand{\prim}[1]{\uline{#1}}
\newcommand{\foreign}[1]{\#\textsl{#1}}

\title{Conversion en relationnel\\du MCD \emph{protected}}
\author{\emph{Généré par Mocodo}}

\begin{document}
\maketitle

\begin{itemize}
  \item \relat{Agence} (\prim{id. agence}, \attr{nom agence})
  \begin{itemize}
    \item Le champ \emph{id. agence} constitue la clé primaire de la table. C'était déjà un identifiant de l'entité \emph{Agence}.
    \item Le champ \emph{nom agence} était déjà un simple attribut de l'entité \emph{Agence}.
  \end{itemize}

  \item \relat{Direction régionale} (\prim{id. dir.}, \attr{nom dir.})
  \begin{itemize}
    \item Le champ \emph{id. dir.} constitue la clé primaire de la table. C'était déjà un identifiant de l'entité \emph{Direction régionale}.
    \item Le champ \emph{nom dir.} était déjà un simple attribut de l'entité \emph{Direction régionale}.
  \end{itemize}

  \item \relat{Superviser} (\foreign{\prim{id. agence}}, \foreign{id. dir.!})
  \begin{itemize}
    \item \paragraph{Avertissement.} Table résultant de la conversion forcée d'une association DF.
    \item Le champ \emph{id. agence} constitue la clé primaire de la table. C'est une clé étrangère qui a migré directement à partir de l'entité \emph{Agence}.
    \item Le champ à saisie obligatoire \emph{id. dir.} est une clé étrangère. Il a migré directement à partir de l'entité \emph{Direction régionale} en perdant son caractère identifiant.
  \end{itemize}

\end{itemize}

\end{document}
