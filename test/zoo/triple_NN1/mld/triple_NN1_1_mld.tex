% Generated by Mocodo 4.0.2

\documentclass[a4paper]{article}
\usepackage[normalem]{ulem}
\usepackage[T1]{fontenc}
\usepackage[french]{babel}
\frenchsetup{StandardLayout=true}

\newcommand{\relat}[1]{\textsc{#1}}
\newcommand{\attr}[1]{#1}
\newcommand{\prim}[1]{\uline{#1}}
\newcommand{\foreign}[1]{\#\textsl{#1}}

\title{Conversion en relationnel\\du MCD \emph{triple_NN1}}
\author{\emph{Généré par Mocodo}}

\begin{document}
\maketitle

\begin{itemize}
  \item \relat{Gérer} (\foreign{\prim{ingénieur}}, \foreign{\prim{projet}}, \foreign{responsable!})
  \begin{itemize}
    \item Le champ \emph{ingénieur} fait partie de la clé primaire de la table. C'est une clé étrangère qui a migré directement à partir de l'entité \emph{Ingénieur}.
    \item Le champ \emph{projet} fait partie de la clé primaire de la table. C'est une clé étrangère qui a migré directement à partir de l'entité \emph{Projet}.
    \item Le champ à saisie obligatoire \emph{responsable} est une clé étrangère. Il a migré directement à partir de l'entité \emph{Responsable} en perdant son caractère identifiant.
  \end{itemize}

  \item \relat{Ingénieur} (\prim{ingénieur}, \attr{nom ingénieur})
  \begin{itemize}
    \item Le champ \emph{ingénieur} constitue la clé primaire de la table. C'était déjà un identifiant de l'entité \emph{Ingénieur}.
    \item Le champ \emph{nom ingénieur} était déjà un simple attribut de l'entité \emph{Ingénieur}.
  \end{itemize}

  \item \relat{Projet} (\prim{projet}, \attr{libellé projet})
  \begin{itemize}
    \item Le champ \emph{projet} constitue la clé primaire de la table. C'était déjà un identifiant de l'entité \emph{Projet}.
    \item Le champ \emph{libellé projet} était déjà un simple attribut de l'entité \emph{Projet}.
  \end{itemize}

  \item \relat{Responsable} (\prim{responsable}, \attr{nom responsable})
  \begin{itemize}
    \item Le champ \emph{responsable} constitue la clé primaire de la table. C'était déjà un identifiant de l'entité \emph{Responsable}.
    \item Le champ \emph{nom responsable} était déjà un simple attribut de l'entité \emph{Responsable}.
  \end{itemize}

\end{itemize}

\end{document}
