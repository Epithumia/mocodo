% Generated by Mocodo 4.0.3

\documentclass[a4paper]{article}
\usepackage[normalem]{ulem}
\usepackage[T1]{fontenc}
\usepackage[french]{babel}
\frenchsetup{StandardLayout=true}

\newcommand{\relat}[1]{\textsc{#1}}
\newcommand{\attr}[1]{#1}
\newcommand{\prim}[1]{\uline{#1}}
\newcommand{\foreign}[1]{\#\textsl{#1}}

\title{Conversion en relationnel\\du MCD \emph{weak}}
\author{\emph{Généré par Mocodo}}

\begin{document}
\maketitle

\begin{itemize}
  \item \relat{Appartement} (\foreign{\prim{code rue}}, \foreign{\prim{num immeuble}}, \foreign{\prim{num étage}}, \prim{num appart.}, \attr{nb pièces})
  \begin{itemize}
    \item Le champ \emph{code rue}, \emph{num immeuble}, \emph{num étage} fait partie de la clé primaire de la table. C'est une clé étrangère qui a migré à partir de l'entité \emph{Étage} pour renforcer l'identifiant.
    \item Le champ \emph{num appart.} fait partie de la clé primaire de la table. C'était déjà un identifiant de l'entité \emph{Appartement}.
    \item Le champ \emph{nb pièces} était déjà un simple attribut de l'entité \emph{Appartement}.
  \end{itemize}

  \item \relat{Immeuble} (\foreign{\prim{code rue}}, \prim{num immeuble}, \attr{nb étages})
  \begin{itemize}
    \item Le champ \emph{code rue} fait partie de la clé primaire de la table. C'est une clé étrangère qui a migré à partir de l'entité \emph{Rue} pour renforcer l'identifiant.
    \item Le champ \emph{num immeuble} fait partie de la clé primaire de la table. C'était déjà un identifiant de l'entité \emph{Immeuble}.
    \item Le champ \emph{nb étages} était déjà un simple attribut de l'entité \emph{Immeuble}.
  \end{itemize}

  \item \relat{Rue} (\prim{code rue}, \attr{nom rue})
  \begin{itemize}
    \item Le champ \emph{code rue} constitue la clé primaire de la table. C'était déjà un identifiant de l'entité \emph{Rue}.
    \item Le champ \emph{nom rue} était déjà un simple attribut de l'entité \emph{Rue}.
  \end{itemize}

  \item \relat{Étage} (\foreign{\prim{code rue}}, \foreign{\prim{num immeuble}}, \prim{num étage}, \attr{nb appartements})
  \begin{itemize}
    \item Le champ \emph{code rue}, \emph{num immeuble} fait partie de la clé primaire de la table. C'est une clé étrangère qui a migré à partir de l'entité \emph{Immeuble} pour renforcer l'identifiant.
    \item Le champ \emph{num étage} fait partie de la clé primaire de la table. C'était déjà un identifiant de l'entité \emph{Étage}.
    \item Le champ \emph{nb appartements} était déjà un simple attribut de l'entité \emph{Étage}.
  \end{itemize}

\end{itemize}

\end{document}
