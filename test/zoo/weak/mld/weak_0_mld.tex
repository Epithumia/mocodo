\documentclass[a4paper]{article}
\usepackage[normalem]{ulem}
\usepackage[T1]{fontenc}
\usepackage[french]{babel}
\frenchsetup{StandardLayout=true}

\newcommand{\relat}[1]{\textsc{#1}}
\newcommand{\attr}[1]{#1}
\newcommand{\prim}[1]{\uline{#1}}
\newcommand{\foreign}[1]{\#\textsl{#1}}

\title{Conversion en relationnel\\du MCD \emph{weak}}
\author{\emph{Généré par Mocodo}}

\begin{document}
\maketitle

\begin{itemize}
  \item \relat{Exemplaire} (\prim{œuvre}, \prim{exemplaire}, \attr{foobar})
  \begin{itemize}
    \item Le champ \emph{œuvre} fait partie de la clé primaire de la table. Il a migré à partir de l'entité \emph{Œuvre} pour renforcer l'identifiant. Cependant, comme la table créée à partir de cette entité a été supprimée, il n'est pas considéré comme clé étrangère.
    \item Le champ \emph{exemplaire} fait partie de la clé primaire de la table. C'était déjà un identifiant de l'entité \emph{Exemplaire}.
    \item Le champ \emph{foobar} a migré à partir de l'association de dépendance fonctionnelle \emph{DF}.
  \end{itemize}

\end{itemize}

\end{document}
