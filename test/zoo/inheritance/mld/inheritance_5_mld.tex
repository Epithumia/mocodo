% Generated by Mocodo 4.0.1

\documentclass[a4paper]{article}
\usepackage[normalem]{ulem}
\usepackage[T1]{fontenc}
\usepackage[french]{babel}
\frenchsetup{StandardLayout=true}

\newcommand{\relat}[1]{\textsc{#1}}
\newcommand{\attr}[1]{#1}
\newcommand{\prim}[1]{\uline{#1}}
\newcommand{\foreign}[1]{\#\textsl{#1}}

\title{Conversion en relationnel\\du MCD \emph{inheritance}}
\author{\emph{Généré par Mocodo}}

\begin{document}
\maketitle

\begin{itemize}
  \item \relat{LACUS} (\prim{magna}, \attr{vestibulum}, \attr{tempor}, \attr{fugit})
  \begin{itemize}
    \item Le champ \emph{magna} constitue la clé primaire de la table. Il était clé primaire de l'entité-mère \emph{TRISTIS} (supprimée).
    \item Le champ \emph{vestibulum} est un simple attribut. Il était simple attribut de l'entité-mère \emph{TRISTIS} (supprimée).
    \item Les champs \emph{tempor} et \emph{fugit} étaient déjà de simples attributs de l'entité \emph{LACUS}.
  \end{itemize}

  \item \relat{NEC} (\prim{magna}, \attr{vestibulum}, \attr{pulvinar}, \attr{audis}, \foreign{magna via_mollis!}, \foreign{magna via_vitae!})
  \begin{itemize}
    \item Le champ \emph{magna} constitue la clé primaire de la table. Il était clé primaire de l'entité-mère \emph{TRISTIS} (supprimée).
    \item Le champ \emph{vestibulum} est un simple attribut. Il était simple attribut de l'entité-mère \emph{TRISTIS} (supprimée).
    \item Les champs \emph{pulvinar} et \emph{audis} étaient déjà de simples attributs de l'entité \emph{NEC}.
    \item Le champ à saisie obligatoire \emph{magna via_mollis} est une clé étrangère. Il a migré par l'association de dépendance fonctionnelle \emph{MOLLIS} à partir de l'entité \emph{LACUS} en perdant son caractère identifiant.
    \item Le champ à saisie obligatoire \emph{magna via_vitae} est une clé étrangère. Il a migré par l'association de dépendance fonctionnelle \emph{VITAE} à partir de l'entité \emph{SODALES} en perdant son caractère identifiant.
  \end{itemize}

  \item \relat{SODALES} (\prim{magna}, \attr{vestibulum}, \attr{convallis}, \attr{ipsum})
  \begin{itemize}
    \item Le champ \emph{magna} constitue la clé primaire de la table. Il était clé primaire de l'entité-mère \emph{TRISTIS} (supprimée).
    \item Le champ \emph{vestibulum} est un simple attribut. Il était simple attribut de l'entité-mère \emph{TRISTIS} (supprimée).
    \item Les champs \emph{convallis} et \emph{ipsum} étaient déjà de simples attributs de l'entité \emph{SODALES}.
  \end{itemize}

  \item \relat{ULTRICES} (\foreign{\prim{magna sodales}}, \foreign{\prim{magna lacus}})
  \begin{itemize}
    \item Le champ \emph{magna sodales} fait partie de la clé primaire de la table. C'est une clé étrangère qui a migré directement à partir de l'entité \emph{SODALES}.
    \item Le champ \emph{magna lacus} fait partie de la clé primaire de la table. C'est une clé étrangère qui a migré directement à partir de l'entité \emph{LACUS}.
  \end{itemize}

\end{itemize}

\end{document}
