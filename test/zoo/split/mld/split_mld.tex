% Generated by Mocodo 4.0.1

\documentclass[a4paper]{article}
\usepackage[normalem]{ulem}
\usepackage[T1]{fontenc}
\usepackage[french]{babel}
\frenchsetup{StandardLayout=true}

\newcommand{\relat}[1]{\textsc{#1}}
\newcommand{\attr}[1]{#1}
\newcommand{\prim}[1]{\uline{#1}}
\newcommand{\foreign}[1]{\#\textsl{#1}}

\title{Conversion en relationnel\\du MCD \emph{split}}
\author{\emph{Généré par Mocodo}}

\begin{document}
\maketitle

\begin{itemize}
  \item \relat{Bataille} (\prim{nom bataille}, \attr{lieu}, \attr{date})
  \begin{itemize}
    \item Le champ \emph{nom bataille} constitue la clé primaire de la table. C'était déjà un identifiant de l'entité \emph{Bataille}.
    \item Les champs \emph{lieu} et \emph{date} étaient déjà de simples attributs de l'entité \emph{Bataille}.
  \end{itemize}

  \item \relat{Trophée} (\prim{numéro}, \attr{type}, \attr{état}, \foreign{nom villageois!}, \foreign{nom bataille!})
  \begin{itemize}
    \item Le champ \emph{numéro} constitue la clé primaire de la table. C'était déjà un identifiant de l'entité \emph{Trophée}.
    \item Les champs \emph{type} et \emph{état} étaient déjà de simples attributs de l'entité \emph{Trophée}.
    \item Le champ à saisie obligatoire \emph{nom villageois} est une clé étrangère. Il a migré par l'association de dépendance fonctionnelle \emph{Récolter} à partir de l'entité \emph{Villageois} en perdant son caractère identifiant.
    \item Le champ à saisie obligatoire \emph{nom bataille} est une clé étrangère. Il a migré par l'association de dépendance fonctionnelle \emph{Récolter} à partir de l'entité \emph{Bataille} en perdant son caractère identifiant.
  \end{itemize}

  \item \relat{Villageois} (\prim{nom villageois}, \attr{adresse}, \attr{fonction})
  \begin{itemize}
    \item Le champ \emph{nom villageois} constitue la clé primaire de la table. C'était déjà un identifiant de l'entité \emph{Villageois}.
    \item Les champs \emph{adresse} et \emph{fonction} étaient déjà de simples attributs de l'entité \emph{Villageois}.
  \end{itemize}

\end{itemize}

\end{document}
