% Generated by Mocodo 4.0.0

\documentclass[a4paper]{article}
\usepackage[normalem]{ulem}
\usepackage[T1]{fontenc}
\usepackage[french]{babel}
\frenchsetup{StandardLayout=true}

\newcommand{\relat}[1]{\textsc{#1}}
\newcommand{\attr}[1]{#1}
\newcommand{\prim}[1]{\uline{#1}}
\newcommand{\foreign}[1]{\#\textsl{#1}}

\title{Conversion en relationnel\\du MCD \emph{triple_N11}}
\author{\emph{Généré par Mocodo}}

\begin{document}
\maketitle

\begin{itemize}
  \item \relat{Affecter} (\foreign{\prim{projet}}, \foreign{\prim{employé}}$^{u_1}$, \foreign{site}$^{u_1}$)
  \begin{itemize}
    \item Le champ \emph{projet} fait partie de la clé primaire de la table. C'est une clé étrangère qui a migré directement à partir de l'entité \emph{Projet}.
    \item Le champ \emph{employé} fait partie de la clé primaire de la table. C'est une clé étrangère qui a migré directement à partir de l'entité \emph{Employé}. Il obéit en outre à la contrainte d'unicité 1.
    \item Le champ \emph{site} est une clé étrangère. Il a migré directement à partir de l'entité \emph{Site} en perdant son caractère identifiant. Il obéit en outre à la contrainte d'unicité 1.
  \end{itemize}

  \item \relat{Employé} (\prim{employé}, \attr{nom employé})
  \begin{itemize}
    \item Le champ \emph{employé} constitue la clé primaire de la table. C'était déjà un identifiant de l'entité \emph{Employé}.
    \item Le champ \emph{nom employé} était déjà un simple attribut de l'entité \emph{Employé}.
  \end{itemize}

  \item \relat{Projet} (\prim{projet}, \attr{libellé})
  \begin{itemize}
    \item Le champ \emph{projet} constitue la clé primaire de la table. C'était déjà un identifiant de l'entité \emph{Projet}.
    \item Le champ \emph{libellé} était déjà un simple attribut de l'entité \emph{Projet}.
  \end{itemize}

  \item \relat{Site} (\prim{site}, \attr{position})
  \begin{itemize}
    \item Le champ \emph{site} constitue la clé primaire de la table. C'était déjà un identifiant de l'entité \emph{Site}.
    \item Le champ \emph{position} était déjà un simple attribut de l'entité \emph{Site}.
  \end{itemize}

\end{itemize}

\end{document}
