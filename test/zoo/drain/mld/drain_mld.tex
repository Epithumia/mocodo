\documentclass[a4paper]{article}
\usepackage[normalem]{ulem}
\usepackage[T1]{fontenc}
\usepackage[french]{babel}
\frenchsetup{StandardLayout=true}

\newcommand{\relat}[1]{\textsc{#1}}
\newcommand{\attr}[1]{#1}
\newcommand{\prim}[1]{\uline{#1}}
\newcommand{\foreign}[1]{\#\textsl{#1}}

\title{Conversion en relationnel\\du MCD \emph{drain}}
\author{\emph{Généré par Mocodo}}

\begin{document}
\maketitle

\begin{itemize}
  \item \relat{Entreprise} (\prim{nom entreprise}, \attr{adresse}, \attr{téléphone})
  \begin{itemize}
    \item Le champ \emph{nom entreprise} constitue la clé primaire de la table. C'était déjà un identifiant de l'entité \emph{Entreprise}.
    \item Les champs \emph{adresse} et \emph{téléphone} étaient déjà de simples attributs de l'entité \emph{Entreprise}.
  \end{itemize}

  \item \relat{Stage} (\prim{num. stage}, \attr{sujet}, \foreign{nom entreprise!}, \attr{date proposition})
  \begin{itemize}
    \item Le champ \emph{num. stage} constitue la clé primaire de la table. C'était déjà un identifiant de l'entité \emph{Stage}.
    \item Le champ \emph{sujet} était déjà un simple attribut de l'entité \emph{Stage}.
    \item Le champ à saisie obligatoire \emph{nom entreprise} est une clé étrangère. Il a migré par l'association de dépendance fonctionnelle \emph{Proposer} à partir de l'entité \emph{Entreprise} en perdant son caractère identifiant.
    \item Le champ \emph{date proposition} a migré à partir de l'association de dépendance fonctionnelle \emph{Proposer}.
  \end{itemize}

  \item \relat{Étudiant} (\prim{num étudiant}, \attr{nom}, \foreign{num. stage}$^{u_1}$, \attr{date signature}, \attr{date?}, \attr{note stage})
  \begin{itemize}
    \item Le champ \emph{num étudiant} constitue la clé primaire de la table. C'était déjà un identifiant de l'entité \emph{Étudiant}.
    \item Le champ \emph{nom} était déjà un simple attribut de l'entité \emph{Étudiant}.
    \item Le champ \emph{num. stage} est une clé étrangère. Il a migré par l'association de dépendance fonctionnelle \emph{Attribuer} à partir de l'entité \emph{Stage} en perdant son caractère identifiant. Il obéit en outre à la contrainte d'unicité 1.
    \item Le champ \emph{date signature} a migré à partir de l'association de dépendance fonctionnelle \emph{Attribuer}.
    \item Le champ à saisie facultative \emph{date} est un simple attribut. Il a migré par l'association de dépendance fonctionnelle \emph{Soutenir} à partir de l'entité \emph{Date} en perdant son caractère identifiant. Cependant, comme la table créée à partir de cette entité a été supprimée, il n'est pas considéré comme clé étrangère.
    \item Le champ \emph{note stage} a migré à partir de l'association de dépendance fonctionnelle \emph{Soutenir}.
  \end{itemize}

\end{itemize}

\end{document}
