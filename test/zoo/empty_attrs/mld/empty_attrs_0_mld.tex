\documentclass[a4paper]{article}
\usepackage[normalem]{ulem}
\usepackage[T1]{fontenc}
\usepackage[french]{babel}
\frenchsetup{StandardLayout=true}

\newcommand{\relat}[1]{\textsc{#1}}
\newcommand{\attr}[1]{#1}
\newcommand{\prim}[1]{\uline{#1}}
\newcommand{\foreign}[1]{\#\textsl{#1}}

\title{Conversion en relationnel\\du MCD \emph{empty_attrs}}
\author{\emph{Généré par Mocodo}}

\begin{document}
\maketitle

\begin{itemize}
  \item \relat{CLIENT} (\prim{Réf. client}, \attr{</span>, <span class='normal'>.1}, \attr{.2})
  \begin{itemize}
    \item Le champ \emph{Réf. client} constitue la clé primaire de la table. C'était déjà un identifiant de l'entité \emph{CLIENT}.
    \item Les champs \emph{</i>, <i>.1} et \emph{.2} étaient déjà de simples attributs de l'entité \emph{CLIENT}.
  \end{itemize}

\end{itemize}

\end{document}
